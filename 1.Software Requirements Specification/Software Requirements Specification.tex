\documentclass[a4]{article}

\author{Group 7, System Black}
\title{Software Requirements Specification}

\begin{document}
\maketitle

\section{Basic Information}
\subsection{Basic Requirements (May be removed later)}
Score Management Subsystem
The subsystem is mainly for teachers to record student achievement and for students to query their scores, and has some statistic analysis functions about the scores.

Score record: after the course exam, teachers can input all students' score, and before formal submission, the teacher can modify and query any scores. After submission, if the teacher
wants to modify any score, he must submit an application form to explain the reason.

Score query: if teachers have submit the course score, students can query his scores, and can get his score's statistic analysis.

Score analysis: this function is mainly for teachers to analyze a course's score distribution, including the average score, score distribution, score ranking statistic. And the result should
show in some form of chart for them to understand better. At the same time the system can show students’ individual achievement in a statistical analysis form including GPA, average scores, total credits and so on.

\subsection{Members}
(alphabetical)

Qi, ZHU

Shijia, WEI

Shiyi, ZHU

Tao, LIN

Ye, QI

\section{Introduction}

\section{Description}

\subsection{User Scenarios}
Ye
\subsection{Data Flow Diagram}
Tao
\subsection{State Diagrams}
Shiyi
\subsection{Class Diagrams}
Shijia
\subsection{CRC Cards}
Qi

\section{Validation Criteria}

\subsection{Functions for Teachers}
\subsubsection{Record Scores}

\subsubsection{Vote Modification}

\subsubsection{Query/Analysis Scores}
A teacher can view score information of all courses that he or she instructs and the scores of which are committed. The score information includes course names, student names, raw scores and grade points.


\subsection{Functions for Students}
\subsubsection{Query/Analysis Scores}
A student can view score information of all courses that he or she has registered and the score of which has been committed by the teacher. The score information includes course names, raw scores and grade points.


A student can view his or her own grade point average (GPA), average score, total credits.



\end{document}


